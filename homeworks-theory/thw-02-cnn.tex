\documentclass[12pt,fleqn]{article}

\usepackage{vkCourseML}

\usepackage{lipsum}
\usepackage{indentfirst}
\usepackage{enumitem}
\usepackage{listings, lstautogobble}
\usepackage{float}
\usepackage{xcolor}

\definecolor{codegray}{rgb}{0.3,0.3,0.3}
\definecolor{lightgray}{rgb}{0.5,0.5,0.5}

\lstdefinestyle{mystyle}{
    deletekeywords={eval},
    keywordstyle=\ttfamily\footnotesize\color{codegray},
    basicstyle=\ttfamily\footnotesize\color{codegray},
    numberstyle=\tiny\color{lightgray},
    breakatwhitespace=false,         
    breaklines=true,                 
    captionpos=b,                    
    keepspaces=true,                 
    numbersep=5pt,                  
    showspaces=false,                
    showstringspaces=false,
    showtabs=false,                  
    tabsize=2,
}

\lstset{style=mystyle}
\lstset{emph={None, False, True, with, for, in},
    emphstyle={\ttfamily\footnotesize\bf\color{black}}%
}

\title{Глубинное обучение 1, ФКН ВШЭ\\Теоретическое домашнее задание №2\\Сверточные нейронные сети}
\author{}
\date{}
\theorembodyfont{\rmfamily}
\newtheorem{esProblem}{Задача}

\begin{document}

\maketitle

\begin{esProblem}
    Посчитайте результат применения операции свертки с ядром K к матрице X. Параметры свертки следующие: stride=2, dilation=2, padding=1 (паддинг осуществляется нулями).
    $$
    X = \begin{bmatrix}
        1 & 0 & -4 & 2 \\
        5 & 2 & 3 & 0 \\
        -1 & 0 & 1 & 4 \\
        0 & -3 & 2 & -1
    \end{bmatrix}
    \quad\quad\quad
    K = \begin{bmatrix}
        2 & 1 \\
        -1 & -2 
    \end{bmatrix}
    $$
\end{esProblem}

\begin{esProblem}
    Посчитайте число обучаемых параметров в нейронной сети, архитектура которой выглядит следующим образом:
    \begin{center}
        \begin{lstlisting}[language=Python]
        model = nn.Sequential(
            nn.Conv2d(
                in_channels=3, out_channels=16, kernel_size=5,
                stride=2, padding=0, dilation=1, bias=True
            ),
            nn.BatchNorm2d(num_features=16),
            nn.LeakyReLU(0.1),
            nn.Conv2d(
                in_channels=16, out_channels=32, kernel_size=5,
                stride=1, padding=1, dilation=2, bias=False
            ),
            nn.BatchNorm2d(num_features=32),
            nn.Sigmoid(),
        )
        \end{lstlisting}
    \end{center}
    
    \vspace{-1.5\baselineskip}
    \noindent
    Сколько параметров будет у этой же нейронной сети, если все 2D свертки заменить на комбинацию depthwise и pointwise сверток?
\end{esProblem}

\end{document}

